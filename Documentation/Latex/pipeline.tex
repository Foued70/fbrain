\section{The BTK pipelines}
BTK allows to implement a pipeline for the processing of fetal images,
i.e. the reconstruction of anatomical and diffusion data, and the final
tractography, all expressed in the same local coordinate system. This
processing can be summarized in the following steps: 1) image conversion, 2) anatomical image reconstruction, 3)
recontruction of the diffusion sequence, 4) registration of diffusion to anatomical data and 5)
tractography.

\subsection{Image conversion}
BTK supports and has been tested by using images in Nifti format
(http://nifti.nimh.nih.gov/nifti-1). However, images are frequently available in
DICOM format and an image conversion is required. This can be performed by using
dcm2nii (http://www.cabiatl.com/mricro/mricron/dcm2nii.html), Slicer or other softwares.

Let say that you have 3 (possibly orthogonal) anatomical images:
\begin{verbatim}
ana01.nii.gz 
ana02.nii.gz 
ana03.nii.gz 
\end{verbatim}

and one set of DWI images: 
\begin{verbatim}
dwi.nii.gz
dwi.bvec 
dwi.bval 
\end{verbatim}

Please check that images are not flipped.\\

\underline{NOTE}: The set of three files .nii.gz, .bvec, and .bval used to
describe a DW sequences are represented in BTK just by the basename (``dwi'' in
the previous example), and the sequences must be provided in this way to the
different applications. This allows to have shorter command lines and the use
of consistent filenames.

\subsection{Anatomical image reconstruction}


  \begin{figure*}[!h]
   \centering{
     \begin{tabular}{c}
         \psfig{figure=overview_anatomical_pipeline.eps,width=14cm}\\
       \end{tabular}
       }
   \caption{Overview of the processing pipeline for anatomical data in BTK. Slicer3D (www.slicer.org) is used to convert the DICOM data to NIFTI format and for the placement of landmarks on which is based the reorientation of the 3D reconstructed image. The use of Slicer for these steps allows the user to check the orientation consistency between the three anatomical images (axial, coronal, sagittal for instance). ITKSNAP (www.itksnap.org) is used to create a rough mask of the brain (this step is optional).}
   \label{fig:anatomical_pipeline}
 \end{figure*}

  \begin{figure*}[!h]
   \centering{
     \begin{tabular}{c}
         \psfig{figure=command_line_anatomical_pipeline.eps,width=14cm}\\
       \end{tabular}
       }
   \caption{Command lines corresponding to the processing pipeline for anatomical data in BTK. On the left are shown the blocks used in the anatomical data pipeline (Figure~\ref{fig:anatomical_pipeline}) and on the right the correspond command lines.}
   \label{fig:anatomical_command_line}
 \end{figure*}
 

The anatomical image reconstruction pipeline is depicted in Figure~\ref{fig:anatomical_pipeline}. The corresponding command lines are shown in Figure~\ref{fig:anatomical_command_line}. This pipeline consists in: 1) converting the data from DICOM to NIFTI, 2) manual masking of the brain (optional step which can improve the reconstruction results) and image cropping (based on the size of the masks), 3) image denoising, 4) motion correction and super-resolution, 5) reorientation of the image using landmarks placed using Slicer. 

\subsection{Reconstruction of the diffusion sequence}

The processing pipeline described in this section is dedicated to DWI of the fetal brain (depicted in Figure~\ref{fig:diffusion_pipeline}). The corresponding command lines are shown in Figure~\ref{fig:diffusion_command_line}. It is similar to the anatomical image pipeline but it includes processing tools dedicated to DWI: 1) data conversion using dcm2nii (www.cabiatl.com/mricro/mricron/dcm2nii.html), 2) sequence normalization (the B0 image are registered together using affine transforms and then averaged to produce one B0 image), 3) B0 image extraction and manual masking using ITKSNAP, 4) motion correction and RBF interpolation to compute the reconstructed DW sequence, 5) reorientation of the image using landmarks placed using Slicer. This final step (sequence reorientation) is very important since it allows the user to visualize the DW data using consistent color coding schemes. Please see also Section \ref{subsec:diff_rec}. 

  \begin{figure*}[!h]
   \centering{
     \begin{tabular}{c}
         \psfig{figure=overview_diffusion_pipeline.eps,width=14cm}\\
       \end{tabular}
       }
   \caption{Overview of the processing pipeline for diffusion data in BTK. dcm2nii is used to convert the DICOM data to NIFTI format and Slicer for the placement of landmarks on which is based the reorientation of the 3D reconstructed image. ITKSNAP (www.itksnap.org) is used to create a rough mask of the brain.}
   \label{fig:diffusion_pipeline}
 \end{figure*}

  \begin{figure*}[!h]
   \centering{
     \begin{tabular}{c}
         \psfig{figure=command_line_diffusion_pipeline.eps,width=14cm}\\
       \end{tabular}
       }
   \caption{Command lines corresponding to the processing pipeline for diffusion data in BTK. On the left are shown the blocks used in the anatomical data pipeline (Figure~\ref{fig:diffusion_pipeline}) and on the right the correspond command lines.}
   \label{fig:diffusion_command_line}
 \end{figure*}

\subsection{Tractography}
If you have followed the previous steps correctly, at this point you should
have the reconstructed anatomical and diffusion data spatially aligned, and
ready to perform the tractography. To do this, BTK provides
\textbf{btkTractography} (Section \ref{subsec:tracto}).
