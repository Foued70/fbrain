\section{The BTK pipeline}
BTK allows to implement a pipeline for the processing of fetal images,
i.e. the reconstruction of anatomical and diffusion data, and the final
tractography, all expressed in the same local coordinate system. This
processing can be summarized in the following steps: 1) image conversion, 2) anatomical image reconstruction, 3)
recontruction of the diffusion sequence, 4) registration of diffusion to anatomical data and 5)
tractography.

\subsection{Image conversion}
BTK supports and has been tested by using images in Nifti format
(http://nifti.nimh.nih.gov/nifti-1). However, images are frequently available in
DICOM format and an image conversion is required. This can be performed by using
dcm2nii (http://www.cabiatl.com/mricro/mricron/dcm2nii.html), Slicer or other softwares.

Let say that you have 3 (possibly orthogonal) anatomical images:
\begin{verbatim}
ana01.nii.gz 
ana02.nii.gz 
ana03.nii.gz 
\end{verbatim}

and one set of DWI images: 
\begin{verbatim}
dwi.nii.gz
dwi.bvec 
dwi.bval 
\end{verbatim}

Please check that images are not flipped.\\

\underline{NOTE}: The set of three files .nii.gz, .bvec, and .bval used to
describe a DW sequences are represented in BTK just by the basename (``dwi'' in
the previous example), and the sequences must be provided in this way to the
different applications. This allows to have shorter command lines and a the use
of consistent filenames.

\subsection{Anatomical image reconstruction}
Anatomical image reconstruction can be performed by using \textbf{btkImageReconstruction} (Section
\ref{subsec:ana_rec})\footnote{Best results are obtained by using a mask for each anatomical image. Such image masks can be easily created using ITKSnap for instance.}, followed by a re-orientation procedure using \textbf{btkSetStandardCoorSystem} and \textbf{btkReorientImageToStandard} (Section \ref{sec:utilities})\footnote{Note that the landmarks file is obtained using external software (Slicer). Please see Section \ref{sec:utilities} for details about this step.}:

\begin{verbatim}
 
btkImageReconstruction -i ana01.nii.gz -i ana02.nii.gz -i ana03.nii.gz 
                       -m ana01_mask.nii.gz -m ana02_mask.nii.gz -m ana03_mask.nii.gz
                       -o ana3D.nii.gz --mask

btkReorientImageToStandard -i ana3D.nii.gz -o ana3D_oriented.nii.gz -l landmarks.fcsv

\end{verbatim}



\subsection{Reconstruction of the diffusion sequence}
To reconstruct diffusion data, you want to follow the steps described in
Section \ref{subsec:diff_rec}. The use of two applications is required here:
\textbf{btkGroupwiseS2SDistortionCorrection} and
\textbf{btkRBFInterpolationS2S}.

\subsection{Registration of diffusion to anatomical data}
This can be performed by using \textbf{btkRegisterDiffusionToAnatomicalData}
(Section \ref{subsec:ana_rec})

\subsection{Tractography}
If you have followed the previous steps correctly, at this point you should
have the reconstructed anatomical and diffusion data spatially aligned, and
ready to perform the tractography. To do this, BTK provides
\textbf{btkTractography} (Section \ref{subsec:tracto}).
